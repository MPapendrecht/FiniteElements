\documentclass[10pt,a4paper]{article}
\usepackage[utf8]{inputenc}
\usepackage[english]{babel}
\usepackage{amsmath}
\usepackage{amsfonts}
\usepackage{amssymb}
\usepackage{graphicx}
\usepackage{hyperref}
\usepackage{epstopdf}
\usepackage{float}
\usepackage{python}
\usepackage{subcaption}

\usepackage{cleveref}
\title{Finite Elements lab assignments}
\author{Martijn Papendrecht: 4369971}
\begin{document}
\maketitle
\newpage
\tableofcontents
\newpage
\section{1D Assignment}
\subsection{Theory}
Starting with the differential equation and boundary conditions
\begin{equation}
\begin{split}
-D \frac{d^2u}{dx^2} + \lambda u = f(x), \\
-D \frac{du}{dx}(0) = 0, -D \frac{du}{dx}(1)=0
\end{split}
\end{equation}
we can derive the weak formulation. 
\begin{equation} 
\label{eq:WeakForm}
\begin{split}
\int_0^1 \left[ -D   \frac{d^2 u}{dx^2} + \lambda u \right] \phi dx &= \int_0^1 f(x) \phi dx \\
\int_0^1  -D \left( \frac{d}{dx} \left( \frac{du}{dx} \phi \right) - \frac{d\phi}{dx} \frac{du}{dx} \right) + \lambda u \phi &= \int_0^1 f(x) \phi dx \\
-D \int_0^1 \frac{d}{dx} \left( \frac{du}{dx} \phi \right) dx + \int_0^1 D \frac{d\phi}{dx}\frac{du}{dx} + \lambda u \phi dx &= \int_0^1 f(x) \phi dx \\
\int_0^1 D \frac{d\phi}{dx} \frac{du}{dx} + \lambda u \phi dx &= \int_0^1 f(x) \phi dx
\end{split}
\end{equation}
Here $\phi$ is the test function satisfying the smoothness requirements. 
The boundary conditions are already containted in this weak formulation.

We can derive the Galerkin formulation by substitution $\phi = \phi_i$ and $u \approx u^N = \sum_{j=1}^N c_j \phi_j$ in \cref{eq:WeakForm}.

\begin{equation} 
\begin{split}
\int_0^1 D \frac{d\phi_i}{dx} \frac{d}{dx}\left( \sum_{j=1}^N c_j \phi_j \right) + \lambda \left( \sum_{j=1}^N c_j \phi_j \right) \phi_i dx &= \int_0^1 f(x) \phi_i dx \\
\sum_{j=1}^N c_j \int_0^1 D \frac{d \phi_i}{dx} \frac{d \phi_j}{dx} + \lambda \phi_i \phi_j dx &= \int_0^1 f(x) \phi_i dx
\end{split}
\end{equation}

And finally we can write in the form $S \vec{u} = \vec{f}$ where 

\begin{align}
S_{ij} &= \int_0^1 D \frac{d \phi_i}{dx} \frac{d \phi_j}{dx} + \lambda \phi_i \phi_j dx \\
f_i &= \int_0^1 f(x) \phi_i dx
\end{align}

\subsection{Modelling}
All figures are made using Matlab. 
The code is available at Github \footnote{ \url{https://github.com/MPapendrecht/FiniteElements.git}}. 
In these figures the elements aren't equally spaced as prescribed in the assignment. 
Every element has an additional random offset of $\pm \frac{1}{2n}$. 
This has been done to test the code without the assumption of equally spaced vertices.

\def\scale{0.20}

The figures below show the results for different number of vertices $n$. $\lambda = 1$ and $D = 1$ and the function $f(x)=1$
\begin{figure}[H]
	\centering
	\includegraphics[scale=\scale]{"./Pictures/linPlot10".eps}
\end{figure}
\begin{figure}[H]
	\centering
	\includegraphics[scale=\scale]{"./Pictures/linPlot20".eps}
\end{figure}
\begin{figure}[H]
	\centering
	\includegraphics[scale=\scale]{"./Pictures/linPlot40".eps}
\end{figure}
\begin{figure}[H]
	\centering
	\includegraphics[scale=\scale]{"./Pictures/linPlot80".eps}
\end{figure}
\begin{figure}[H]
	\centering
	\includegraphics[scale=\scale]{"./Pictures/linPlot160".eps}
\end{figure}

The solution is as expected very close to the analytical solution, and clearly satisfies the boundary conditions. 

The figures below show the results for different number of vertices $n$ as well. The same parameters are used, however the function $f(x)=sin(20x)$.

\begin{figure}[H]
	\centering
	\includegraphics[scale=\scale]{"./Pictures/sinPlot10".eps}
\end{figure}
\begin{figure}[H]
	\centering
	\includegraphics[scale=\scale]{"./Pictures/sinPlot20".eps}
\end{figure}
\begin{figure}[H]
	\centering
	\includegraphics[scale=\scale]{"./Pictures/sinPlot40".eps}
\end{figure}
\begin{figure}[H]
	\centering
	\includegraphics[scale=\scale]{"./Pictures/sinPlot80".eps}
\end{figure}
\begin{figure}[H]
	\centering
	\includegraphics[scale=\scale]{"./Pictures/sinPlot160".eps}
\end{figure}

The solution with $n=10$ shows the boundary conditions aren't satisfied properly because the amount of vertices is too small. 
The other solutions satisfy the boundary conditions much better. 
Increasing $n$ further doesn't seem to make the function improve much more, which can be seen in the figure below.
\begin{figure}[H]
	\centering
	\includegraphics[scale=\scale]{"./Pictures/sinPlot1600".eps}
\end{figure}

\newpage
\section{2D Assignments}
\subsection{Assignment 1}

\end{document}
